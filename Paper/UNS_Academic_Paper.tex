
\documentclass[11pt]{amsart}
\usepackage{amsmath, amssymb, amsthm}
\usepackage{hyperref}

\title{The Universal Number Set (UNS):\\
A Representation-Invariant Numerical Framework}
\author{Reed Kimble}
\date{\today}

\begin{document}
\maketitle

\begin{abstract}
We introduce the \emph{Universal Number Set (UNS)}, a numerical framework in which
numbers are defined as microstate-distributed functions with state-dependent evaluation,
representation invariance, and total operational closure. UNS embeds classical arithmetic
while extending it with expressive capabilities not available in traditional number systems.
This paper formalizes UNS, positions it within theoretical mathematics, compares it to
classical numerical frameworks, and provides philosophical justification for its structure.
\end{abstract}

\section{Introduction}

Classical mathematics treats numbers as atomic objects: elements of sets such as
$\mathbb{Z}, \mathbb{Q}, \mathbb{R}$, or $\mathbb{C}$. These systems offer well-defined
algebraic and analytic tools but lack means to express internal distribution, undefined
operations, or representation-invariant semantics. The \emph{Universal Number Set (UNS)}
proposes a structural generalization in which a number is a function
$f : X \to \mathbb{C}$ over a normalized microstate domain $(X,\mu)$.
This reinterprets classical numbers as constant functions and enables new forms of
computation and analysis.

\section{Definition of UNS}

A UNS universe is a measure space $(X,\mu)$ with $\mu(X)=1$. A \emph{UNS value} is a function
$f : X \to \mathbb{C}$. A \emph{state} is a function $\psi : X \to \mathbb{C}$ satisfying
\[
\int_X |\psi(x)|^2\, d\mu(x) = 1.
\]
The \emph{readout} of a value $f$ under a state $\psi$ is defined by
\[
\mathrm{read}(f \mid \psi)
= \int_X f(x)\, |\psi(x)|^2\, d\mu(x).
\]

\section{Dimensional Invariance}

UNS requires readout to be invariant under any admissible dimensional transform
$D(N,\psi)$, a structural refinement that preserves normalization and functional mapping.
This requirement has no classical analogue and resembles coordinate-free structures in
geometry rather than traditional number systems.

\section{Total Operational Closure}

Classical arithmetic is partial: operations such as $a/0$ are undefined. UNS replaces this
with \emph{novel values}, symbolic but first-class objects with provenance:
\[
\frac{3}{0} \mapsto \mathrm{novel}(\mathrm{divide}, (3,0), x).
\]
This treatment differs from non-standard analysis, domain theory, and error algebra
frameworks. Instead, UNS asserts that arithmetic should be total, compositional, and
structurally meaningful.

\section{Comparisons with Classical and Extended Number Systems}

UNS intersects thematically with probability theory, Hilbert space methods, operator
algebras, and non-standard numbers, but does not coincide with any of them. Unlike
hyperreals, surreals, or $p$-adics, UNS extends not the \emph{set of scalar values} but the
\emph{structure of what a number is}. Classical numbers appear as constant functions, while
UNS values may encode variation across microstates.

\section{Computational Semantics}

A canonical execution model is defined using Q16.16 fixed-point complex values, finite
microstate arrays, lifted unary and binary operations, and provenance-aware novel value
tables. This makes UNS directly executable on embedded systems, web runtimes, and
general-purpose architectures. Most mathematical frameworks do not specify such a
canonical computational semantics.

\section{Philosophical Significance}

UNS aligns with structuralist and representation-independent mathematics
\cite{Shapiro1997, Landry2007}. It reframes numbers as invariant structures rather than
atomic elements, while retaining compatibility with classical arithmetic. UNS suggests that
numeric identity may not be inherently atomic, but instead distributed and evaluative.

\section{Conclusion}

UNS constitutes a novel numerical category: representation-invariant, operationally total,
distribution-sensitive, and computationally grounded. It embeds classical arithmetic while
extending expressive capacity in ways unavailable to existing number systems.

\bibliographystyle{amsplain}
\bibliography{uns_refs}

\end{document}
